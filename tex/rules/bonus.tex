% -*- Mode: LaTeX -*-
%        File: bonus.tex
%      Author: Bryan Jurish <moocow@ling.uni-potsdam.de>
% Description: know rules: bonus rules
%
%---------------------------------------------------------------------------
% \section{Bonus Rules}\label{bonus}
%---------------------------------------------------------------------------
\subsection{Conservation of Randomness}\label{conserve}
  The outcome of a game of \know\ often depends largely on chance,
  as manifested by die rolls.  For this reason (among others), the
  principle of {\sl Conservation of Randomness} was introduced.
  By this principle, every die roll cast in the course of a game
  of \know\ {\bf must} be moved by some piece.  Some situations
  where Conservation of Randomness has been applied include
  the following:
  \begin{itemize}
    \item
      On casting dice to determine which player will initially
      begin the game, the initiating player does not re-roll
      the dice for his or her initial move, but rather moves the
      rolls as cast.

    \item
      When a {\bf Solipsistic Path} is attempted and fails
      (case \ref{sopath_nope}), the contender does not re-roll
      the dice for his or move, but rather moves the rolls as
      cast for the attempted solipsistic path.

    \item
      For the Eternal Version, if a player achieves three connections
      during post-resolution movement with a die roll left unmoved,
      this roll must be moved by the player who first achieved
      resolution in his or her initial turn of the following cycle.
  \end{itemize}

\subsection{Nets}\label{nets}
  Two pieces are said to be in a {\bf Net Relation} to one another
  when the following conditions apply:
  \begin{enumerate}
    \item The pieces are of the same type
    \item The pieces occupy correspondent spaces on different quadrants of the board.
  \end{enumerate}

  In other words, a  Net Relation between two pieces is very similar to a
  connection relation, except that the Netted pieces need not be members
  of the same color-alignment, and that the Netted members need not be on opposite
  {\bf sides} of the board (as with connections), but rather only on different
  quadrants. For each space on the board, there are exactly three other
  spaces on the board in which a sibling piece, regardless of alignment,
  might stand to produce a ``netted'' relation.  (Recall that an alignment
  is a group of six pieces as played by a single player in the two-player
  version of the game.)

  A {\bf Polar Net} occurs when the Netted pieces are in adjacent quadrants: when
  a Connection relation could {\sl not} exist between the spaces occupied by the Netted
  pieces.  A {\bf Balanced Net} occurs when the Netted pieces are not in adjacent
  quadrants: when a Connection relation {\sl could} exist between the spaces occupied
  by the netted pieces.

  The {\bf Duration} of a Net is the span of time during which at least one piece of
  the Net Type (the type of the Netted pieces - bards, fools, or heralds) finds itself
  in a species of Net Relation (polar or balanced) with some other piece of the Net
  Type at the beginning or end of a piece's move, similar to Connections.

\subsubsection{The Conduit and the Conversation}\label{conduit}

  A Netted piece may act as a {\bf Conduit} for another piece of the same type, to
  allow its sibling piece to perform a type of move known as a {\bf Conversation}.
  A Conduit is said to be Polar or Balanced, depending on the flavor of Net in
  which it is participating.

  The Conversation bears some resemblance to the Solipsistic Path or the Significant
  Errand, in that it can allow a piece to move through spaces influenced by a 
  piece of the other alignment.  A Conversation differs from a Solipsistic Path or
  Significant Errand, however, in that no die rolls are required in order to
  determine whether or not such a move is possible.  A Conversation is said to be
  {\bf two-sided} if the piece acting as Conduit is a member of a Balanced Net,
  {\bf one-sided} otherwise.

  A One-Sided Conversation allows free movement by pieces of the Net Type 
  through spaces influenced by any pieces of the Net Type, for the duration of the
  Conversation.  Effectively, all spaces influenced by pieces of
  the Net Type are ``turned off'' for all pieces of that type
  for the duration of a balanced net.

  A Two-Sided Conversation allows all pieces of the Net Type to move freely through
  spaces influenced by any pieces {\sl not} of the Net Type, for the duration of the
  Conversation, i.e. as long as there is a Balanced Net relation between two of the
  Net-type pieces.

  Note that for pieces not of the Net Type, spaces continue to be influenced as
  usual.

%%% Local Variables: 
%%% mode: latex
%%% TeX-master: "~/text/projects/know/tex/know"
%%% End: 
