% -*- Mode: LaTeX -*-
%        File: layout.tex
%      Author: Bryan Jurish <moocow@ling.uni-potsdam.de>
% Description: know rules: board-layout
%
%---------------------------------------------------------------------------
% \section{Concerning the Layout of the Playing Board}\label{layout}
%---------------------------------------------------------------------------

\know\ is played on a board constructed from sixteen points
evenly distributed about the circumference of a circle by
connecting certain pairs of points.  \appref{board} contains
a version of the \know\ board.

%---------------------------------------------------------------------------
\subsection{Spaces and Borders}\label{spaces}

  The \know\ board is divided into exactly one hundred {\bf spaces}, each
  of which {\bf borders on} a number of spaces of the opposite color.
  A space is said to ``border on'' another space if the two spaces share
  a common {\sl side}.  Spaces sharing only a single common {\sl corner} do not
  border on one another.  In particular, the center of the board itself
  is a corner rather than a space.

  As a consequence of this definition and the coloring conventions
  for spaces, any unfilled (white) space will border only on
  filled (black) spaces, and any filled space will border only on
  unfilled spaces.



%---------------------------------------------------------------------------
\subsection{Dimensions and Directions}\label{directions}

  There are four main {\bf dimensions} (or ``axes'')
  which can be identified on the \know\ board.  In terms of these
  dimensions, eight {\bf directions} may be defined.
  Some of these terms might even be used later in this document.

  The four axes and their associated directions are as follows:

  \begin{itemize}
  \item {\bf Horizontal Axis}
    \begin{itemize}
    \item {\sl Forward}\\
      Proceeding from the player's corner-space to the
      opposing player's corner-space.
    \item {\sl Backward}\\
      Proceeding from the opposing player's
      corner-space to the player's corner-space.
    \end{itemize}
    
  \item {\bf Vertical Axis}\\
    The vertical axis is (intuitively) perpendicular to the horizontal
    axis.  It is not particularly useful.
    %\begin{itemize}
    %\item {\sl Up}
    %\item {\sl Down}
    %\end{itemize}

  \item {\bf Radial Axis}
    \begin{itemize}
      \item {\sl Hubwards}\\
        Proceeding from the outer edge of the board towards
        the center of the board, or ``hub''.
      \item {\sl Rimwards}\\
        Proceeding from the center of the board towards the
        outer edge, or ``rim''.
    \end{itemize}

  \item {\bf Rotational Axis}
    \begin{itemize}
      \item {\sl Turnwise}\\
        Proceeding from left to right while remaining a fixed distance
        from the hub -- also known as ``clockwise''.
      \item {\sl Widdershins}\\
        Proceeding from right to left while remaining a fixed distance
        from the hub -- also known as ``counter-clockwise''.
    \end{itemize}
  \end{itemize}

The directions listed above are not always mutually exclusive --
for example, one movement may be understood as proceeding in both a
forward and turnwise direction.  The differences between the various
directions can be understood as stemming from the distinction between
player-relative directions (such as ``forward'') and board-absolute
directions (such as ``hubwards'').

%---------------------------------------------------------------------------
\subsection{Tiers}\label{tiers}

  The board is divided into four {\bf Tiers} (or ``levels''), each of which
  constitutes an implied square on the playing board. 
  The descriptions in the rest of this section refer to
  \figref{tiers_fig}\latex{ on page \pageref{tiers_fig}}.

  %
    \begin{figure}
      \begin{center}
        % lengths & commands for key
        \newlength{\tierboxh}
        \setlength{\tierboxh}{1em}
        \newcommand{\tierbox}[1]{%
          \raisebox{-3pt}{\includegraphics*[height=\tierboxh]{pics/#1-box-\bwcolor.\pspdf}}%
        }
        % key
        \makeimage{%
          \latex{
            \begin{tabular}{|rl|}
              \hline
              \multicolumn{2}{|c|}{\bf Key}\\
              Center Tier: & \tierbox{center}\\
              Second Tier: & \tierbox{tier2}\\
              First Tier: & \tierbox{tier1}\\
              Rim/First Ambiguous: & \tierbox{ambig}\\
              Rim: & \tierbox{rim}\\
              \hline
            \end{tabular}
            \par
            % figure
            \includegraphics*[width=0.9\textwidth]{pics/tiers-\bwcolor.\pspdf}%
          }
        }
        \begin{rawhtml}
          <img src="tiers-key.png">
          <br>
          <img src="tiers.png">
        \end{rawhtml}
        \caption{An Illustration of the Game's Four Tiers}
        \label{tiers_fig}
      \end{center}
    \end{figure}


\subsubsection{Center Tier}
A space is said to be in the {\bf Center Tier} if the
center of the board is one of its corners.  The Center Tier
contains sixteen spaces.

\subsubsection{Second Tier}
Any space which is not in the Center Tier, but which shares
all of its corners with some space sharing at least one
corner with a space in the Center Tier is said to be
in the {\bf Second Tier}.

Formally, for a space $S$, $S$ is in the Second Tier
if and only if for every point $p$ which is a corner
of $S$, there exists a point $p^\prime$ and spaces
$S_p$ and $S_p^\prime$ such that $p$ is a corner
of $S_p$, $p^\prime$ is a corner shared by both
$S_p$ and $S_p^\prime$,
and $S_p^\prime$ is in the Center Tier.
In most cases, $S_p = S$.

The Second Tier contains thirty-two spaces.

\subsubsection{First Tier}
A space is said to be in the {\bf First Tier} if it is neither
in the Center Tier nor in the Second Tier, and if it shares at
least one corner with some space in the Second Tier.  The First Tier
contains thirty-two spaces.

\subsubsection{Rim}
A space is said to be a member of the {\bf Rim Tier}, or ``on the Rim''
if one of its borders is an edge of the board itself.
The Rim contains twenty-four spaces, four of which also belong
to the First Tier.  These four triangular spaces are known as
``ambiguous'' spaces.



%%% Local Variables: 
%%% mode: latex
%%% TeX-master: "~/text/projects/know/tex/know"
%%% End: 
