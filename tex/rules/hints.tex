% -*- Mode: LaTeX -*-
%        File: hints.tex
%      Author: Bryan Jurish <moocow@ling.uni-potsdam.de>
% Description: know rules: versions
%
%---------------------------------------------------------------------------
% \section{Some Hints}\label{hints}
%---------------------------------------------------------------------------

\subsection{Threads}\label{threads}
  It is often helpful to think of two pieces of the same color and stripe
  (pieces which can make connections) as attached by an invisible
  ``thread'' running through the center of the board.  Ideally, the angle of
  this ``thread'' at the center of the board should be brought closer to
  180$^\circ$ (a straight line) by a movement of one or both of these
  pieces, since such an angle is a prerequisite for a connection, and
  thus for resolution.  At the beginning of the game, this generally
  means that moving ``forward'' (towards the opposing player's starting
  positions) is desirable.

\subsection{Walls}\label{walls}
  A common strategy is for a player (or color) to build a ``wall'' accross the board
  with his or her pieces and their influences.  Doing so will typically hinder the other
  player from moving one piece of a given type to the opposite side of the board, in
  effect ``isolating'' two pieces of the same type on one half of the board, so that no
  connections between those pieces are possible unless one of them is to move through
  a point of contention.  In this case, the player with the isolated pieces should
  probably consider using a Solipsistic Path to get one of his or her
  pieces through to the opposite side of the board.

%%% Local Variables: 
%%% mode: latex
%%% TeX-master: "~/text/projects/know/tex/know"
%%% End: 
