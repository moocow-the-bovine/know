% -*- Mode: LaTeX -*-
%        File: movement.tex
%      Author: Bryan Jurish <moocow@ling.uni-potsdam.de>
% Description: know rules: movement
%
%---------------------------------------------------------------------------
% \section{Concerning Movement}\label{movement}
%---------------------------------------------------------------------------

\subsection{Dice}\label{dice}
The standard version of \know\ requires at least one six-sided (cubical)
die to play.

\subsection{Turns}\label{turns}
  \subsubsection{Initiation}\label{turn_begin}
  A player's turn begins when he or she receives the die (or dice) from
  the opposing player.
  \subsubsection{Conclusion}\label{turn_end}
  A player's turn has officially ended when he or she removes his or her
  dice from the board or equivalent rolling surface.

\subsection{Rolling}\label{rolling}
In some special cases,
such as the 
Significant Errand
or the Solipsistic Path
a player must announce his or her intention to invoke
an exceptional movement rule before the die is cast.
Normally, however, a player may roll as soon as his or her
turn begins.  See \secref{move_except} for details on
exceptional movement rules.

\subsection{Jumps}\label{jumps}
For each die rolled, a player must move exactly one piece.
Some variats of \know\ require the player to roll more than one
die per turn, but in this Section I will assume that each player
rolls exactly one six-sided die per turn, and consequently moves
exactly one piece per turn.

The piece which is chosen to be moved must make exactly as many
{\bf jumps} as the die roll indicates.  From any space on the board,
a piece may ``jump'' once to any of the spaces which
border\footnote{See \secref{spaces} for a definition of ``borders''}
on the space it is currently occupying.

Since bordering spaces are necessarily of opposite color
to the space occupied, it is often more important for determining
a move whether the die roll is even or odd.  With an even die roll,
the piece moved must upon ending its move occupy a space of the
same color as the space on which it began its move.  With an odd
die roll, the piece moved must occupy a space of the opposite color
at the end of its move.

\subsection{Redundant Paths}\label{redundant}
A piece may not occupy the same space more than once in the course
of one movement.  This means that no space may be jumped into or out
of more than once in one movement.  In other words, a piece cannot
go back on its own path in one turn, nor can a piece make a loop on
the board, ending its movement by jumping into the same space
which it occupied at the beginning of that turn.  Such
movements are known as {\sl redundant paths}.

\subsection{Mandatory Movement}\label{mandatory}
If any piece can be jumped the exact number of spaces indicated by a
player's die roll, then some move must be made -- a player cannot
choose to ``pass'' his or her turn, but must move some piece if at
all possible.



%---------------------------------------------------------------------------
% Influence
%---------------------------------------------------------------------------
\subsection{Influence}\label{influence}

A piece is said to {\bf influence} every space bordering on that space which
it currently occupies.  No piece of another player may jump either into or
through such an {\bf influenced space},
except in the exceptional cases of 
the Exceptional Movement Rules, as described in
\secref{move_except}.

Under no conditions may a piece jump either into or through a
space occupied by another piece, even if that piece belongs to
the same player.  This restriction applies even to moves
falling under the Exceptional Movement Rules of \secref{move_except}.

\subsubsection{Points of Contention}\label{contention}
A {\bf point of contention} is a space on the board influenced by two
pieces of opposite color.  Movement through points of contention is
possible by means of the Exceptional Movement Rules described in
\secref{move_except}, but is otherwise not allowed.

Under no conditions may any piece occupy a point of contention
at the end of its movement.


%---------------------------------------------------------------------------
% Exceptional Movements
%---------------------------------------------------------------------------
\subsection{Exceptional Movements}\label{move_except}

%---------------------------------------------------------------------------
% Significant Errand
%---------------------------------------------------------------------------
\subsubsection{The Significant Errand}\label{errand}

If a piece is entirely surrounded by points of contention -- that is, if there are no moves
possible for that piece, it is said to be ``locked out''.  In this case, the player
belonging to that piece may, when his or her turn comes, roll the
die (or dice, as the case may be) and request
a {\bf Significant Errand}, based on his or her die roll for
that turn.  Resolution of Significant Errands proceeds as follows:

  \begin{enumerate}
  \item
    The player with the locked out piece rolls the die
    as usual at the start of his or her turn.

  \item
    If the roll is such that the locked out piece could move exactly
    that number of spaces, disregarding all points of contention and
    spaces influenced by pieces of the opposing color, the player
    with the locked out piece may at this point request a
    {\bf Significant Errand}.

  \item
    The opposing player rolls the die.

  \item
    If the opposing player's die roll is greater than the locked
    out player's initial roll, the locked out player must move
    some other piece, with the normal rules regarding points of
    contention and influenced spaces still applying.

  \item
    If the locked out player's roll is higher, then he or she must
    move his or her locked-out piece, provided that the piece, on completing its move
    of the player's initial die roll, finds itself in a space not
    controlled by the opposing player, and not on the same space as
    any other piece.
  \end{enumerate}


%---------------------------------------------------------------------------
% Solipsistic Path
%---------------------------------------------------------------------------
\subsubsection{The Solipsistic Path}\label{sopath}

It often happens that at least one color will manage to build a
``wall'' of influenced spaces across one and/or both sides of the board,
effectively preventing the other player from moving his or her pieces
through to the opposite side
(which is necessary to achieve Resolution -- see \secref{connectres}).

Although it is possible to play entire games without invoking this rule, if it is
entirely impossible for pieces to move through points of contention, then there
is also the possibility that the game will never be Resolved.  So, if a player
wants very much for one of his/her/its/their beloved pieces to reach the opposite
half of the board, and if the opposing player has built a wall of influence
across the board such that the piece which is to be sent on its way must pass
through at least one point of contention to reach its intended destination,
then the player who wishes to venture across the point of contention may
invoke the {\bf Solipsistic Path}, by the following procedure:

  \begin{enumerate}
  \item
    The player who wishes to pass through the Point of Contention declares
    their intention to do so before they roll the die, and point out which
    space is being contended.

    The Point of Contention being challenged must have existed for at least
    one roll of both players' dice before either player may invoke the
    {\bf Solipsistic Path}.  The rationale here is that each player should
    have had a chance to ``back down'' before this rule is invoked.

  \item
    The mover (the player of the piece which is attempting to move
    through a Point of Contention) and the contender (the player whose
    piece(s) is/are blocking the mover's intended path) roll one
    six-sided die each.

  \item\label{sp_tally}
    Both the mover and the contender count the number of spaces influenced
    by any of their pieces whose influence also extends over the Point of
    Contention to be traversed, discounting any points of contention among
    those influenced spaces, and add these numbers to their die rolls.

  \item
    If the contender's tally from Step \ref{sp_tally} is greater than the
    mover's tally, then the mover's turn is forfeit: none of the mover's
    pieces may be moved on this turn, and the die gets passed to the contender.

  \item
    If the tallies are equal, the players should roll again until one player's
    tally is higher.

  \item\label{sopath_nope}
    If the mover's tally is greater than the contender's, then the mover must move
    the piece for which the Solipsistic Path was declared exactly the number of spaces
    indicated by his/her die roll, (ignoring the addition of influenced spaces which
    were used to determine the outcome of the Solipsistic Path), moving first into the
    Point of Contention which was challenged, and passing through any number of points
    of contention, provided that the piece, on finishing its move, occupies a space which
    is not influenced by a piece of the opposite color, or occupied by any other piece.

  \item\label{sopath_noso}
    If the mover's tally was greater, but movement of the Solipsistic Piece was
    impossible (i.e. there was no possible destination under that die roll which was not
    influenced by a piece of the contender's), then
    the mover's turn is forfeit.
    %the mover must move some other piece on that turn.
  \end{enumerate}

%%% Local Variables: 
%%% mode: latex
%%% TeX-master: "~/text/projects/know/tex/know"
%%% End: 
