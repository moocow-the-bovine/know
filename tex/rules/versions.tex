% -*- Mode: LaTeX -*-
%        File: versions.tex
%      Author: Bryan Jurish <moocow@cudmuncher.de>
% Description: know rules: versions
%
%---------------------------------------------------------------------------
% \section{Versions of the Game}\label{versions}
%---------------------------------------------------------------------------

\subsection{Two Player Versions}\label{2player}

\subsubsection{Standard Version}\label{2player_std}
  In the standard two-player version of \know, each player adopts one color of pieces and one
  side of the board as his or her own.  Each player rolls one six-sided die per turn, and moves
  exactly one piece exactly the number of spaces indicated by his or her die roll per turn.  This
  is the basic version of the game, for which most of the movement rules in this document
  were formulated.

\subsubsection{Symmetric Version}\label{2player_sym}
  In this version of \know, each player rolls two six-sided dice per turn, and must
  move two separate pieces: exactly one piece per die roll (except in the cases of
  Resolution and the Solipsistic Path, of course).  The two pieces moved on one
  turn must be of the same type -- i.e. a player must move two Bards, or two Fools,
  or two Heralds, but not a Bard and a Herald.

  The pieces moved in the course of one turn may be of differing types on one
  condition only: if the first piece moved makes a connection (not necessarily a
  Resolution-type connection -- any connection will do) with that move, then
  the player may move a different type of piece for the second die roll.

  A player may move only one piece per turn only in the case of a Solipsistic Path,
  for which the rules do not allow movement of more than one piece.

  In the case of a Significant Errand, which a player may declare after he or she
  rolls the dice, the player with the locked out piece must choose which die roll
  he or she wishes to use to contend the lockout before the opposing player rolls
  (alternately, of course, an opposing player may choose to allow free passage
  through any Point of Contention at any time), and before the moving player moves
  any pieces.

  This \know\ variant progresses faster than the ``standard'' version
  described above, and happens to be the author's personal favorite.

\subsection{Four Player Versions}\label{4player}

  The four-player versions correspond directly to the two-player versions,
  except for the fact that in a four-player version, each player is assigned
  three pieces instead of six: one of each type, and the {\bf players} of the
  two-player version (i.e. the two colors of pieces) are replaced by
  {\bf Conjunctons} of two players.  Each player is responsible for
  controls one Bard, one Herald, and one Fool of a given
  color + stripe combination.

  In the Standard Version, play proceeds turnwise (counter-clockwise) around
  the board, with no restrictions on piece-type movement.

  In the Symmetric Version, players of one Conjunction roll their dice simultaneously, and
  both must move the same type of piece, each moving exactly his or her die roll.
  Players of a single color alternate conjunction-initial movements.

\subsection{Three Player Versions}\label{3player}

In the three-player version of \know, each player is assigned exactly one
type of piece.  Initial placement differs from that shown in
\figref{iplace_fig}.  Resolution is achieved when a player's pieces
are connected on both the First and Second Tiers.


\subsection{Eternal Versions}\label{eternal}
  For two- and four-player games, \know\ can be made
  eternal\footnote{
    at least, as eternal as the players are willing to put up with
    }.
  Resolution (followed by post-resolution, and scoring if desired)
  ends not the game itself, but only a {\sl cycle}
  of the game.

  The initial placement for the next cycle is then achieved by
  exchanging all of the striped pieces on the board: bards for
  bards, fools for fools, and heralds for heralds.

  The player who first achieved Resolution is the first to move
  in the following cycle.

  %When playing such a (semi-)eternal game, it can happen that
  %a round begins in which some piece stands in a point of
  %contention.  No special movement rules apply to such pieces.
 

%%% Local Variables: 
%%% mode: latex
%%% TeX-master: "~/text/projects/know/tex/know"
%%% End: 
