% -*- Mode: LaTeX -*-
%        File: pieces.tex
%      Author: Bryan Jurish <moocow@ling.uni-potsdam.de>
% Description: know rules: pieces
%
%---------------------------------------------------------------------------
% \section{Concerning the Playing Pieces}\label{pieces}
%---------------------------------------------------------------------------

\subsection{Naming Conventions}\label{pnames}
Twelve pieces are required for a game of \know, six per ``alignment''
(or ``color'' or ``side'').
The two sets of six pieces are divided into three types, so
that each alignment (i.e. each player in a two-player game) has two
representatives of each of the three classes of game piece.
The three classes of game piece are:

\newlength{\pieceheight}
\setlength{\pieceheight}{1.5em}

\begin{itemize}
\item
  {\bf Bards}
  \hfill
  \latex{
    \includegraphics*[height=\pieceheight]{pics/bard1-\bwcolor.\pspdf}
    \includegraphics*[height=\pieceheight]{pics/bard2-\bwcolor.\pspdf}
    }
  \begin{rawhtml}
    <img src="bard1.png">
    <img src="bard2.png">
  \end{rawhtml}
  \\
  A less wealthy mode of nobility,
  Bards are the pieces with the strongest initial position.
  They are represented as cups, spheres, or unmarked stones.

  \item
    {\bf Fools}
    \hfill
    \latex{
      \includegraphics*[height=\pieceheight]{pics/fool1-\bwcolor.\pspdf}
      \includegraphics*[height=\pieceheight]{pics/fool2-\bwcolor.\pspdf}
      }
    \begin{rawhtml}
      <img src="fool1.png">
      <img src="fool2.png">
    \end{rawhtml}
    \\
    High-ranking clergy of general silliness,
    Fools are represented as bells, cones, or stones marked
    with a zero (``0'').

  \item
    {\bf Heralds}
    \hfill
    \latex{
      \includegraphics*[height=\pieceheight]{pics/herald1-\bwcolor.\pspdf}
      \includegraphics*[height=\pieceheight]{pics/herald2-\bwcolor.\pspdf}
      }
    \begin{rawhtml}
      <img src="herald1.png">
      <img src="herald2.png">
    \end{rawhtml}
    \\
    Non-violent victims of the medieval military establishment,
    Heralds are represented as flags, cylinders, or stones marked
    with a one (``1'').
\end{itemize}

The two pieces of the same type and color may be distinguished from
one another by the presence or absence of a {\sl stripe}.  That is,
each player has the following set of pieces:
\begin{itemize}
  \item Bard (striped)
  \item Bard (unstriped)
  \item Fool (striped)
  \item Fool (unstriped)
  \item Herald (striped)
  \item Herald (unstriped)
\end{itemize}

\subsection{Initial Placement}\label{iplacemt}

At the opening of each game, each player's pieces should be arranged into two
main {\bf clusters}, each of which consists of one Bard, one Fool,
and one Herald, and which should be arranged symmetrically on the Rim Tier
nearest the player as shown in \figref{iplace_fig}
\latex{on page \pageref{iplace_fig}}.  Player 1's pieces
are shown in {\bf \acolor}, and Player 2's pieces are shown in
{\bf \bcolor}.

\begin{figure}
  \begin{makeimage}
    \latex{\includegraphics*[width=\textwidth]{pics/newgame-\bwcolor.\pspdf}}
  \end{makeimage}
  \begin{rawhtml}
    <center><img align="center" src="newgame.png"></center>
  \end{rawhtml}
  \caption{Initial Placement of Pieces}
  \label{iplace_fig}
\end{figure}

Bards begin the game on the ambiguous spaces.
On the first shaded space behind ({\sl backwards} of) each bard's
position, a fool should be placed.  The Heralds begin the game
on the unshaded Rim spaces behind the Fools' initial positions.


%%% Local Variables: 
%%% mode: latex
%%% TeX-master: "~/text/projects/know/tex/know"
%%% End: 
