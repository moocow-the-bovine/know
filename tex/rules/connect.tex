% -*- Mode: LaTeX -*-
%        File: connect.tex
%      Author: Bryan Jurish <moocow@ling.uni-potsdam.de>
% Description: know rules: connections & resolution
%
%---------------------------------------------------------------------------
%\section{Concerning Connections and Resolution}\label{connectres}
%---------------------------------------------------------------------------

%---------------------------------------------------------------------------
% Connections
%---------------------------------------------------------------------------
\subsection{Connections}\label{connect}

Two pieces are said to be {\bf connected} if the following
conditions all apply to the positions of those pieces on the board:

\begin{enumerate}
  \item
    The pieces occupy ``corresponding spaces'' on opposite sides of the
    center of the board.

  \item
    The pieces are the of the same {\bf type} -- only
    Fools can be connected with Fools, only
    Bards with Bards, etc.

  \item
    The pieces belong to the same player.
\end{enumerate}

\subsubsection{Correspondence}\label{corresp}
The board contains exactly fifty pairs of {\sl corresponding spaces}.
Two spaces are said to ``correspond'' if and only if:
\begin{enumerate}
  \item
    At least one side of each piece's occupied space is
    collinear with exactly one line which passes through the
    center of the board.

  \item
    The occupied spaces are on the same Tier of the board.
\end{enumerate}

Phrased differently, two spaces correspond if they 
are mirror-images separated by 180 degrees.  See
\figref{connect_fig} for an example.


\begin{figure}
  \makeimage{%
    \latex{%
      \includegraphics*[width=\textwidth]{pics/connect-\bwcolor.\pspdf}%
      }%
  }
  \begin{rawhtml}
    <center><img align="center" src="connect.png"></center>
  \end{rawhtml}
  \caption{Resolving and Non-Resolving Connections}
  \label{connect_fig}
\end{figure}


%---------------------------------------------------------------------------
% Resolution
%---------------------------------------------------------------------------
\subsection{Resolution}\label{resolution}
To make and maintain connections on each of Rim, First Tier, and Second Tier
is the primary goal of a game of \know.
The player who first achieves all three of these simultaneous connections is said
to have {\bf Resolved} the game.

For beginners, the suggested rules for determing if a pattern of three connections
is a {\bf resolved} pattern are simply these: a resolved pattern must consist of
one connection on Rim, one connection on First Tier, and one connection on
Second Tier.  Connections on the Center Tier cannot be resolving connections.

The following discussion of {\sl Connections} and {\sl Resolving Connections}
refers to \figref{connect_fig}\latex{ on page \pageref{connect_fig}}.


In \figref{connect_fig}, most of the pieces shown are {\sl connected},
except for the \bcolor\ Bards.  Only the \bcolor\ Bards are not connected because, although
there is a certain symmetry between the spaces they occupy, the line of symmetry does
not pass through the center of the board.  In terms of the formal definition of a connection,
they are not connected because none of the line-segments which form any of their spaces'
sides are, in fact, on the same line themselves.  In terms of the less formal definition
of connections, the \bcolor\ Bards are separated by less than 180 degrees on the
board.

The \acolor\ pieces in \figref{connect_fig} show a Resolving pattern:
the Fools are connected on the Rim,
the Bards are connected on the First Tier, and the Heralds are connected on the
Second Tier.  Since the \bcolor\ Bards are not connected, \bcolor\ cannot be
Resolved.  Nonetheless, the \bcolor\ player could achieve resolution by
establishing a connection between his or her Bards on either Rim or on First
Tier, since the \bcolor\ Heralds are connected on the Second Tier, and the
\bcolor\ Fools are connected on ambiguous spaces.\footnote{
  See \secref{layout} for details on ambiguous spaces.
  }


%---------------------------------------------------------------------------
% Post-Resolution
%---------------------------------------------------------------------------
\subsection{Post-Resolution}\label{postres}

Once one of the players has Resolved a game (or round), his or her pieces may
not move again in the course of that game/round.  Since the player who resolves
the game may well owe this fact to pure chance, the non-resolved player receives
a post-resolution chance to complete his or her own resolution pattern.
The procedure for post-resolution play, in which only the player who is not resolved
moves his or her pieces, is as follows:

\begin{enumerate}
  \item
    If the Unresolved player already has three connections at Resolution, then the
    game (or round) is ended.  Otherwise, if the Unresolved player has fewer than
    three connections, he or she should roll the die as usual, immediately following
    Resolution.

  \item
    If it is possible to move any currently unconnected piece the exact number
    of spaces indicated by the die roll, allowing movement through but not into
    any Points of Contention and make a connection, the Unresolved player may do
    so.

  \item
    If the Unresolved player made a connection on the Post-Resolution roll, then he or she
    may roll the die again, moving another as yet unconnected piece according to the
    above Post-Resolution movement rules, earning another die roll if a connection
    is made on that move.

  \item
    This process continues until the Unresolved player either
    cannot make a connection with the die roll he or she received, or until the
    Unresolved player has three connections.
\end{enumerate}

%---------------------------------------------------------------------------
% Scoring
%---------------------------------------------------------------------------
\subsubsection{Scoring}\label{scoring}
  To determine the winner of the game (if you must), each player should tally up
  the number of spaces influenced by connected pieces of his or her color, excluding
  any points of contention from this tally, and any player whose pieces are in a Resolved
  pattern adds an extra six influence points for the balanced pattern of three connections
  between six pieces he or she has achieved.  The rationale behind the six-point
  {\sl Resolution Bonus} is that player
  with a Resolved pattern receives influence points not only for the spaces his or her
  pieces influence, but also for the spaces they occupy.

  Note that spaces influenced by pieces which are not connected are not added
  to this tally.

   Note also that the tallies should reflect the number of {\bf spaces} influenced by pieces
   of that player, which is not necessarily the sum of the number of spaces influenced by
   all of that player's pieces.  If, for instance, a player's connections were such that
   more than one of his or her connected pieces influenced the same space, that player
   would add that space only once to his or her tally, since the tally is of total spaces
   influenced.  Of course, if this same space were also a point of
   contention, neither player would not be allowed to add it to his or her tally at all.


\endinput

The formal rules regarding the pieces which may be used for Resolving Connections
and other conditions for these connections are as follows:

\begin{itemize}
  \item Fools may have a Resolving Connection on either First or Second Tier.

  \item Heralds may have a Resolving Connection on either on Rim or First Tier.

  \item Bards may have a Resolving Connection on any Tier except Center.

  \item No pieces may have resolution-type connections on Ambiguous or Center spaces.

  \item
    The Resolving Rim Connection must be on a non-ambiguous space.

  \item
    The Resolving First Tier Connection must be either on a black space, in the case
    that the Resolving Rim Connection is on a non-ambiguous white space; or on a white
    space, in the case that the Resolving Rim Connection is on a non-corner black space.
    All that is important for this rule is this:  a player's Resolving First Tier Connection
    must be on a space which is of opposite color to that player's Resolving Rim Connection.

  \item
    The Resolving Second Tier Connection must be on the same color space as the
    player's Resolving Rim Connection, and must not be on an ambiguous space.
\end{itemize}

%%% Local Variables: 
%%% mode: latex
%%% TeX-master: "~/text/projects/know/tex/know"
%%% End: 
